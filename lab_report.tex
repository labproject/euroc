%Lab Report
\documentclass[12pt,titlepage, a4paper]{article}
 \usepackage[german]{babel}

\begin{document}

\title{Report\\LAB Kognitive Robotik}

\author{M\"unstermann,  Cedrick\\  Krombach, Nicola\\[1cm]
	Autonomous Intelligent Systems\\ \textsc{Universit\"at Bonn}\\}



\maketitle


\section*{Abstract}


\section{Einleitung}
Im Rahmen der Projektgruppe Kognitive Robotik sollten in diesem Jahr die Aufgaben der European Robotics Challenges\footnote{http://www.euroc-project.eu/} bearbeitet werden. 
Die Aufgaben der European Robotics Challenges unterteilen sich dabei in drei Challenges, die wiederum verschiedene Unteraufgaben haben:

\begin{itemize}
 \item Challenge 1: Station\"are Manipulationsroboter in Kooperation mit Menschen (Track 1 \& 2)
 \item Challenge 2: Mobile Manipulationsroboter f\"ur die Logistik (Track 1 \& 2)
 \item Challenge 3: Flugroboter f\"ur industrielle Inspektion (Track 1 \& 2)
\end{itemize}

blabla mehr zu euroc? \\
Unsere Gruppe befasste sich mit dem ersten Track der dritten Challenge, bei welchem die Lokalisierung des MAV und die 3D-Rekonstruktion der Umgebung mit Hilfe von Stereo-Bildern
im Vordergrund stand.


\section{Aufgabenstellung} 
\subsection{Task 1 - Visuelle Lokalisierung}
Vision based localization
Localize w.r.t. starting point
Local accuracy/consistency
Real-time computation!

				
\subsection{Task 2 - 3D-Rekonstruktion}

Reconstruct environment in order to create a 3-D occupancy grid
Camera poses are given
create occupancy grid from depth images
Real-time computation, but over the whole dataset



\section{Experimente}




\section{Zusammenfassung}


\end{document}
